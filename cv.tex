%%%%%%%%%%%%%%%%%
% This is an sample CV template created using altacv.cls
% (v1.7.2, 28 August 2024) written by LianTze Lim (liantze@gmail.com). Compiles with pdfLaTeX, XeLaTeX and LuaLaTeX.
%
%% It may be distributed and/or modified under the
%% conditions of the LaTeX Project Public License, either version 1.3
%% of this license or (at your option) any later version.
%% The latest version of this license is in
%%    http://www.latex-project.org/lppl.txt
%% and version 1.3 or later is part of all distributions of LaTeX
%% version 2003/12/01 or later.
%%%%%%%%%%%%%%%%

%% Use the "normalphoto" option if you want a normal photo instead of cropped to a circle
% \documentclass[10pt,a4paper,normalphoto]{altacv}

\documentclass[10pt,a4paper,ragged2e,withhyper]{altacv}
%% AltaCV uses the fontawesome5 and simpleicons packages.
%% See http://texdoc.net/pkg/fontawesome5 and http://texdoc.net/pkg/simpleicons for full list of symbols.

% Change the page layout if you need to
\geometry{left=1.25cm,right=1.25cm,top=1.5cm,bottom=1.5cm,columnsep=1.2cm}

% The paracol package lets you typeset columns of text in parallel
\usepackage{paracol}

% Change the font if you want to, depending on whether
% you're using pdflatex or xelatex/lualatex
% WHEN COMPILING WITH XELATEX PLEASE USE
% xelatex -shell-escape -output-driver="xdvipdfmx -z 0" sample.tex
% \ifxetexorluatex
%   % If using xelatex or lualatex:
%   \setmainfont{Roboto Slab}
%   \setsansfont{Lato}
%   \renewcommand{\familydefault}{\sfdefault}
% \else
  % If using pdflatex:
  \usepackage[rm]{roboto}
  \usepackage[defaultsans]{lato}
  % \usepackage{sourcesanspro}
  \renewcommand{\familydefault}{\sfdefault}
% \fi

% Change the colours if you want to
\definecolor{SlateGrey}{HTML}{2E2E2E}
\definecolor{LightGrey}{HTML}{666666}
\definecolor{DarkPastelRed}{HTML}{450808}
\definecolor{PastelRed}{HTML}{8F0D0D}
\definecolor{GoldenEarth}{HTML}{E7D192}
% ETH Colors
\definecolor{ETHBlue}{HTML}{215CAF}
\definecolor{ETHBlue10}{HTML}{E9EFF7}
\definecolor{ETHBlue20}{HTML}{D3DEEF}
\definecolor{ETHBlue40}{HTML}{A6BEDF}
\definecolor{ETHBlue60}{HTML}{7A9DCF}
\definecolor{ETHBlue80}{HTML}{4D7DBF}
\definecolor{ETHBlue120}{HTML}{08407E}
\definecolor{ETHPetrol}{HTML}{007894}
\definecolor{ETHPetrol10}{HTML}{E7F4F7}
\definecolor{ETHPetrol20}{HTML}{CCE4EA}
\definecolor{ETHPetrol40}{HTML}{99CAD5}
\definecolor{ETHPetrol60}{HTML}{66AFC0}
\definecolor{ETHPetrol80}{HTML}{3395AB}
\definecolor{ETHPetrol120}{HTML}{00596D}
\definecolor{ETHGreen}{HTML}{627313}
\definecolor{ETHGreen10}{HTML}{E0E3D0}
\definecolor{ETHGreen20}{HTML}{E0E3D0}
\definecolor{ETHGreen40}{HTML}{C0C7A1}
\definecolor{ETHGreen60}{HTML}{A1AB71}
\definecolor{ETHGreen80}{HTML}{818F42}
\definecolor{ETHGreen120}{HTML}{365213}
\definecolor{ETHBronze}{HTML}{8E6713}
\definecolor{ETHBronze10}{HTML}{F4F0E7}
\definecolor{ETHBronze20}{HTML}{E8E1D0}
\definecolor{ETHBronze40}{HTML}{D2C2A1}
\definecolor{ETHBronze60}{HTML}{BBA471}
\definecolor{ETHBronze80}{HTML}{A58542}
\definecolor{ETHBronze120}{HTML}{704F12}
\definecolor{ETHRed}{HTML}{B7352D}
\definecolor{ETHRed10}{HTML}{F8EBEA}
\definecolor{ETHRed20}{HTML}{F1D7D5}
\definecolor{ETHRed40}{HTML}{E2AEAB}
\definecolor{ETHRed60}{HTML}{D48681}
\definecolor{ETHRed80}{HTML}{C55D57}
\definecolor{ETHRed120}{HTML}{96272D}
\definecolor{ETHGrey}{HTML}{6F6F6F}
\definecolor{ETHGrey10}{HTML}{F1F1F1}
\definecolor{ETHGrey20}{HTML}{E2E2E2}
\definecolor{ETHGrey40}{HTML}{C5C5C5}
\definecolor{ETHGrey60}{HTML}{A9A9A9}
\definecolor{ETHGrey80}{HTML}{8C8C8C}
\definecolor{ETHGrey120}{HTML}{575757}
% colors of cv
\colorlet{name}{black}
\colorlet{tagline}{ETHBlue80}
\colorlet{heading}{ETHBlue}
\colorlet{headingrule}{ETHBlue80}
\colorlet{subheading}{ETHPetrol}
\colorlet{accent}{ETHRed}
\colorlet{emphasis}{ETHRed}
\colorlet{body}{ETHGrey120}
\colorlet{subsubheading}{ETHRed80}

% Change some fonts, if necessary
\renewcommand{\namefont}{\Huge\bfseries}
\renewcommand{\personalinfofont}{\footnotesize}
\renewcommand{\cvsectionfont}{\LARGE\bfseries}
\renewcommand{\cvsubsectionfont}{\large\bfseries}


% Change the bullets for itemize and rating marker
% for \cvskill if you want to
\renewcommand{\cvItemMarker}{{\small\textbullet}}
\renewcommand{\cvRatingMarker}{\faCircle}
% ...and the markers for the date/location for \cvevent
% \renewcommand{\cvDateMarker}{\faCalendar*[regular]}
% \renewcommand{\cvLocationMarker}{\faMapMarker*}

% highlight command
\newcommand{\highlight}[1]{{\color{ETHBlue120}\textbf{#1}}}

% If your CV/résumé is in a language other than English,
% then you probably want to change these so that when you
% copy-paste from the PDF or run pdftotext, the location
% and date marker icons for \cvevent will paste as correct
% translations. For example Spanish:
% \renewcommand{\locationname}{Ubicación}
% \renewcommand{\datename}{Fecha}


%% Use (and optionally edit if necessary) this .tex if you
%% want to use an author-year reference style like APA(6)
%% for your publication list
% \input{pubs-authoryear.cfg}

%% Use (and optionally edit if necessary) this .tex if you
%% want an originally numerical reference style like IEEE
%% for your publication list
\input{pubs-num.cfg}

%% sample.bib contains your publications
\addbibresource{cv.bib}

\begin{document}
\name{Egor Im}
\tagline{PhD Candidate in Theoretical Physics}
%% You can add multiple photos on the left or right
\photoR{2.8cm}{portfolio_photo_cropped_colour.jpg}
% \photoL{2.5cm}{Yacht_High,Suitcase_High}

\personalinfo{%
  % Not all of these are required!
  \homepage{egorim.win}
  \email{egor.im.97@gmail.com}
  \phone{+41 76 579 45 11}
  %\mailaddress{Winterthurerstrasse 616, 8057 Zürich}
  \location{Z\"urich, Switzerland}
  \\
  % \twitter{@twitterhandle}
  %\xtwitter{@x-handle}
  \github{eg0rim}
  \linkedin{egor-im}
  \orcid{0000-0001-8506-5046}
  %% You can add your own arbitrary detail with
  %% \printinfo{symbol}{detail}[optional hyperlink prefix]
  % \printinfo{\faPaw}{Hey ho!}[https://example.com/]

  %% Or you can declare your own field with
  %% \NewInfoFiled{fieldname}{symbol}[optional hyperlink prefix] and use it:
  % \NewInfoField{gitlab}{\faGitlab}[https://gitlab.com/]
  % \gitlab{your_id}
  %%
  %% For services and platforms like Mastodon where there isn't a
  %% straightforward relation between the user ID/nickname and the hyperlink,
  %% you can use \printinfo directly e.g.
  % \printinfo{\faMastodon}{@username@instace}[https://instance.url/@username]
  %% But if you absolutely want to create new dedicated info fields for
  %% such platforms, then use \NewInfoField* with a star:
  % \NewInfoField*{mastodon}{\faMastodon}
  %% then you can use \mastodon, with TWO arguments where the 2nd argument is
  %% the full hyperlink.
  % \mastodon{@username@instance}{https://instance.url/@username}
}

\makecvheader
%% Depending on your tastes, you may want to make fonts of itemize environments slightly smaller
% \AtBeginEnvironment{itemize}{\small}

%% Set the left/right column width ratio to 6:4.
\columnratio{0.6}

% Start a 2-column paracol. Both the left and right columns will automatically
% break across pages if things get too long.
\begin{paracol}{2}
\cvsection{Experience}

\cvevent{Scientific Assistant}{ETH Z\"urich}{Jun 2021 -- Ongoing}{Z\"urich, Switzerland}
Integral part of PhD program at the Institute for Theoretical Physics.
\begin{itemize}
  \item Conducting research in the field of mathematical physics.
  \item Teaching assistant for a number of theoretical physics courses for both bachelor and master students.
  \item Assisted with the supervision of a master thesis.
\end{itemize}

\cvsection{Projects}

\cvevent{Polylogarithm Functions on Higher-Genus Surfaces}{ETH Z\"urich}{}{}
\vspace{-0.25em}
Study of the polylogarithm functions on higher-genus Riemann surfaces and their 
constructions using Schottky uniformization.
\begin{itemize}
  \item Developed a \highlight{Python package} and a \highlight{C library} to evaluate higher-genus polylogarithms (using openMP, soon to be published).
  \item Worked in a \highlight{successful collaboration} of 5 researchers.
\end{itemize}

\medskip

\cvevent{Algebraic Structures in AdS/CFT Integrability}{ETH Z\"urich}{}{}
\vspace{-0.25em}
Study of Lie bialgebras and quantum algebras in the context of AdS\textsubscript{5}/CFT\textsubscript{4} integrability.
\begin{itemize}
\item Applied abstract algebraic structures to the physical problem.
\item Developed a \highlight{Mathematica package} to work with Lie bialgebras.
\item Conducted an \highlight{independent research} project under supervision of Prof. Niklas Beisert.
\end{itemize}

\cvsection{Education}

\cvevent{PhD in Theoretical Physics}{ETH Z\"urich}{Jun 2021 -- Ongoing}{Z\"urich, Switzerland}
{\color{ETHBlue80}Tentative thesis title:} ``Hopf Algebras in QFT and String Theories''\\
{\color{ETHBlue80}Supervisor:} Prof. Niklas Beisert

\medskip

\cvevent{MSc in Physics}{ETH Z\"urich}{Sep 2019 -- Apr 2021}{Z\"urich, Switzerland}
{\color{ETHBlue80} GPA:} \textbf{5.8/6.0 with distinction}\\
{\color{ETHBlue80}Major fields of study:} \cvtag{theoretical physics} \cvtag{quantum field theory} \cvtag{string theory} \cvtag{integrability} \cvtag{Lie symmetries}
 
\medskip

\cvevent{BSc in Applied Mathematics and Physics}{St Petersburg University}{Sep 2015 -- June 2019}{St Petersburg, Russia}
{\color{ETHBlue80} GPA:} \textbf{5.0/5.0 with distinction}\\
{\color{ETHBlue80}Major fields of study:} \cvtag{computational physics} \cvtag{numerical algorithms} \cvtag{programming}  \cvtag{high performance computations} \cvtag{networks}
 
\newpage

\cvsection{Conferences}
\cvsubsection{\printinfo{\faMicrophone}{Talks}}
\begin{itemize}
  \item Workshop on Representation Theory and Mathematical Physics \href{https://smfi.unipr.it/en/node/100218}{\color{ETHBlue120}``New Perspectives on Yangians and Quantum Affine Algebras''}, Parma, Italy, 1-3 October 2024.
  \item The XXVIII International Conference on Integrable Systems and Quantum Symmetries (\href{http://isqs.eu/}{\color{ETHBlue120}ISQS28}), Prague, Czech Republic, 1-5 July 2024.
\end{itemize}

% \cvsubsection{\printinfo{\faImage}{Posters}}
% \begin{itemize}
%   \item Integrability in Gauge and String Theories (\href{https://exact.ictp-saifr.org/igst-2024/}{\color{ETHBlue120}IGST 2024}), São Paulo, Brazil, 17-21 June 2024.
%   \item Integrability, Dualities and Deformations (\href{https://indico.cern.ch/event/1370523/page/34966-titles-abstracts-and-short-timetable}{\color{ETHBlue120}IDD 2024}), Swansea, UK, 15-19 July 2024.
%   \item Integrability in Gauge and String Theories (\href{https://indico.phys.ethz.ch/event/49/page/25-contributions}{\color{ETHBlue120}IGST 2024}), Zürich, Switzerland, 19-23 June 2023.
% \end{itemize}

%% Switch to the right column. This will now automatically move to the second
%% page if the content is too long.
\switchcolumn

\cvsection{Languages}

\cvskill{English (C1 level)}{5}
\smallskip
\cvskill{German (B1 level)}{2.5}
\smallskip
\cvskill{Russian (native)}{5} %% Supports X.5 values.

\cvsection{Skills}

% Don't overuse these \cvtag boxes — they're just eye-candies and not essential. If something doesn't fit on a single line, it probably works better as part of an itemized list (probably inlined itemized list), or just as a comma-separated list of strengths.

\cvsubsection{Programming languages}
\smallskip
\cvtag{Python} \cvtag{Mathematica} \cvtag{C/C++} \cvtag{Java}\\
\smallskip
\cvsubsection{Tools}
\smallskip
\cvtag{NumPy} \cvtag{Matplotlib} \cvtag{PySide/PyQt} \cvtag{SciPy} \cvtag{MPI/MPI4Py} \cvtag{openMP}
\cvtag{GSL} \cvtag{Git} 
\cvtag{TensorFlow} \cvtag{Pandas}  \cvtag{IntelMKL}\\
\cvtag{Bash} \cvtag{\LaTeX} \cvtag{Copilot} \cvtag{SQLite}

%% Yeah I didn't spend too much time making all the
%% spacing consistent... sorry. Use \smallskip, \medskip,
%% \bigskip, \vspace etc to make adjustments.
\medskip

\cvsection{Certificates}

\cvcertificate{DeepLearning.AI TensorFlow Developer}{Coursera}{Oct 2024}{\href{https://coursera.org/verify/professional-cert/86WCGZW0PODB}{\color{ETHBlue120}86WCGZW0PODB}}
{\color{ETHBlue80}Acquired skills:} \cvtag{deep learning} \cvtag{TensorFlow} \cvtag{computer vision} \cvtag{NLP} \cvtag{time series}

\cvsection{Other Activities}
\cvevent{Organization of IGST 2023}{ETH Z\"urich}{Jun 2023}{Z\"urich, Switzerland}
Participated in organization of the conference ``Integrability in Gauge and String Theories 2023''.
Developed a set of \highlight{Bash scripts} to automate the \href{https://indico.phys.ethz.ch/event/49/page/26-hybrid-setup}{\color{ETHBlue120}hybrid setup} of the conference.

\medskip

\cvevent{Night of Physics 2022}{ETH Z\"urich}{Jun 2022}{Z\"urich, Switzerland}
Participated in the public scientific fair with a stand on the topic of curved spaces. Developed the simulator (\highlight{using C++ and openMP}) to travel through curved 3D spaces \href{https://hypray2.phys.ethz.ch/}{\color{ETHBlue120}hypray2}.

\medskip

\cvevent{Article Management Tool}{}{}{}
\vspace{-0.25em}
{\color{ETHRed}\textbf{GitHub:}} \href{https://github.com/eg0rim/amt}{\color{ETHBlue120}eg0rim/amt}\\
Developer of a tool to manage articles and references for scientific writing. Supports API requests to arxiv.org to fetch metadata and PDFs. Uses \highlight{Python}, \highlight{PySide}, \highlight{SQLite}.

\newpage

\cvsection{Referees}

% \cvref{name}{email}{mailing address}
\cvref{Prof.\ Niklas Beisert}{ETH Z\"urich}{nbeisert@ethz.ch}

\divider

\cvref{Dr.\ Johannes Br\"odel}{ETH Z\"urich}{jbroedel@ethz.ch}


\end{paracol}

\cvsection{Publications}

%% Specify your last name(s) and first name(s) as given in the .bib to automatically bold your own name in the publications list.
%% One caveat: You need to write \bibnamedelima where there's a space in your name for this to work properly; or write \bibnamedelimi if you use initials in the .bib
%% You can specify multiple names, especially if you have changed your name or if you need to highlight multiple authors.
\mynames{Im/Egor}
%% MAKE SURE THERE IS NO SPACE AFTER THE FINAL NAME IN YOUR \mynames LIST

\nocite{*}

% \printbibliography[heading=pubtype,title={\printinfo{\faBook}{Books}},type=book]

% \divider

\printbibliography[heading=pubtype,title={\printinfo{\faFile*[regular]}{Journal Articles}},type=article]

% \divider

% \printbibliography[heading=pubtype,title={\printinfo{\faUsers}{Conference Proceedings}},type=inproceedings]



\end{document}
